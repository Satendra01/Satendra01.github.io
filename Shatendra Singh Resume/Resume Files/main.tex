%%%%%%%%%%%%%%%%%
% This is an sample CV template created using altacv.cls
% (v1.1.4, 27 July 2018) written by LianTze Lim (liantze@gmail.com). Now compiles with pdfLaTeX, XeLaTeX and LuaLaTeX.
% 
%% It may be distributed and/or modified under the
%% conditions of the LaTeX Project Public License, either version 1.3
%% of this license or (at your option) any later version.
%% The latest version of this license is in
%%    http://www.latex-project.org/lppl.txt
%% and version 1.3 or later is part of all distributions of LaTeX
%% version 2003/12/01 or later.
%%%%%%%%%%%%%%%%

%% If you need to pass whatever options to xcolor
\PassOptionsToPackage{dvipsnames}{xcolor}

%% If you are using \orcid or academicons
%% icons, make sure you have the academicons 
%% option here, and compile with XeLaTeX
%% or LuaLaTeX.
% \documentclass[10pt,a4paper,academicons]{altacv}

%% Use the "normalphoto" option if you want a normal photo instead of cropped to a circle
% \documentclass[10pt,a4paper,normalphoto]{altacv}

\documentclass[10pt,a4paper]{altacv}
%% AltaCV uses the fontawesome and academicon fonts
%% and packages. 
%% See texdoc.net/pkg/fontawecome and http://texdoc.net/pkg/academicons for full list of symbols.
%% 
%% Compile with LuaLaTeX for best results. If you
%% want to use XeLaTeX, you may need to install
%% Academicons.ttf in your operating system's font 
%% folder.


% Change the page layout if you need to
\geometry{left=1cm,right=9cm,marginparwidth=6.8cm,marginparsep=1.2cm,top=1.25cm,bottom=1.25cm,footskip=2\baselineskip}

% Change the font if you want to.

% If using pdflatex:
\usepackage[T1]{fontenc}
\usepackage[utf8]{inputenc}
\usepackage[default]{lato}

% If using xelatex or lualatex:
% \setmainfont{Lato}

% Change the colours if you want to
\definecolor{Mulberry}{HTML}{72243D}
\definecolor{SlateGrey}{HTML}{2E2E2E}
\definecolor{LightGrey}{HTML}{666666}
\colorlet{heading}{Sepia}
\colorlet{accent}{Mulberry}
\colorlet{emphasis}{SlateGrey}
\colorlet{body}{LightGrey}

% Change the bullets for itemize and rating marker
% for \cvskill if you want to
\renewcommand{\itemmarker}{{\small\textbullet}}
\renewcommand{\ratingmarker}{\faCircle}
%% sample.bib contains your publications
\addbibresource{sample.bib}

\usepackage[colorlinks]{hyperref}

\begin{document}

\name{SHATENDRA SINGH}
\photo{2.8cm}{newimg}
\personalinfo{%
  % Not all of these are required!
  % You can add your own with \printinfo{symbol}{detail}
  \email{singh.shatendra@gmail.com }
  \phone{+918962525784}
  \location{Madhya Pradesh ,India}
  \homepage{http://people.iiti.ac.in/~ms1904101007/}
  \linkedin{https://www.linkedin.com/in/shatendra-singh-417560195/}
  %% You MUST add the academicons option to \documentclass, then compile with LuaLaTeX or XeLaTeX, if you want to use \orcid or other academicons commands.
%   \orcid{orcid.org/0000-0000-0000-0000}
}

%% Make the header extend all the way to the right, if you want. 
\begin{fullwidth}
\makecvheader
\end{fullwidth}

%% Depending on your tastes, you may want to make fonts of itemize environments slightly smaller
% \AtBeginEnvironment{itemize}{\small}


%% Provide the file name containing the sidebar contents as an optional parameter to \cvsection.
%% You can always just use \marginpar{...} if you do
%% not need to align the top of the contents to any
%% \cvsection title in the "main" bar.
\cvsection[page1sidebar]{Objective}

\begin{quote}
''To be the part of an enthusiastic work environment,
where I can use my technical skills to accomplish 
organizational goals.''
\end{quote}

\cvsection{Education}

\cvevent{MS(Research)\ in CSE }{IIT Indore}{July 2019 -- July 2021}{}
\begin{itemize}
    \item 8.83 CPI
\end{itemize}


\divider

\cvevent{B.E.\ in CSE}{Jabalpur Engineering College}{Aug 2012 -- May 2016}{}
\begin{itemize}
    \item 72.2 \%
\end{itemize}
\divider

\cvevent{Intermediate\ in PCM,CS}{Kendriya Vidyalaya ,Jabalpur}{Apr 2010 -- Mar 2011}{}
\begin{itemize}
    \item 86.6 \%
\end{itemize}
\textcolor{body!30}{\hdashrule{\linewidth}{0.6pt}{0.5ex}}
\cvevent{Matric\ }{Kendriya Vidyalaya ,Jabalpur}{Apr 2009 -- Mar 2010}{}
\begin{itemize}
    \item 88.8 \%
\end{itemize}
\cvsection{Projects}

\cvevent{\begin{itemize}\item Next Word Predictor\end{itemize}}{IIT Indore}{Aug 2019 -- Nov 2019}{}
\begin{itemize}
\item It is a project based on natural language processing.The machine learning model was trained using Word2Vec Model and LSTM networks.
\end{itemize}

\cvevent{\begin{itemize}\item AI Chatbot\end{itemize}}{IIT Indore}{July 2019-Nov 2019}{}
\begin{itemize}
\item A bot takes forward the conversation with the user by predicting the answers for user's queries.The project majorily involves the usage of the following modules/techniques:
Numpy,Tensorflow,Nltk,Bag of words.
\end{itemize}

\cvevent{\begin{itemize}\item Computer Science Department Portal\end{itemize}}{JEC Jabalpur}{July 2015-Dec 2015}{}
\begin{itemize}
\item The web portal was developed using Xampp,PHP,Adobe Dreamweaver,MySQL-2005.
\end{itemize}

\medskip

\clearpage


%% If the NEXT page doesn't start with a \cvsection but you'd
%% still like to add a sidebar, then use this command on THIS
%% page to add it. The optional argument lets you pull up the 
%% sidebar a bit so that it looks aligned with the top of the
%% main column.
% \addnextpagesidebar[-1ex]{page3sidebar}

\end{document}
